\documentclass[11pt]{article}

%%% These are some packages that are useful
\usepackage{circuitikz} %used for circuit diagrams
\usepackage{lastpage} %allows us to determine how many total pages there are
\usepackage{amsfonts, lipsum}
\usepackage{amsmath,amssymb, amscd,amsbsy, amsthm, enumerate}
\usepackage{mdframed, titlesec, setspace,verbatim, multicol}
\usepackage[top=1in, bottom=1in, left=1in, right=1in]{geometry} %sets the margins
\usepackage[unicode]{hyperref} %enhanced references
\usepackage{tikz, pgfplots, xcolor}
\usepackage{fancyhdr} %creates a fancy header for labeling document/name/date 
\usepackage{listings}
\usepackage{xcolor}
\usepackage{vwcol}  
\usepackage{enumitem}

%%% These are some shortcuts that are handy
\def\real{{\mathbb R}}
\def\Natural{\mathbb{N}}
\def\dx{\textnormal{dx}}
\def\dy{\textnormal{dy}}
\def\dz{\textnormal{dz}}
\def\dt{\textnormal{dt}}
\def\ds{\textnormal{ds}}
\def\dw{\textnormal{dw}}
\def\Re{\textnormal{Re}}
\def\Im{\textnormal{Im}}
\def\exp{\textnormal{exp}}
\def\interior{\textnormal{interior}}
\def\al{\alpha}
\def\del{\delta}
\def\Del{\Delta}
\def\gam{\gamma}
\def\Gam{\Gamma}
\def\Om{\Omega}
\def\ep{\varepsilon}
\def\lam{\lambda}
\def\rational{{\mathbb Q}}
\def\integer{{\mathbb Z}}
\def\Q{{\mathbb Q}}
\def\Z{{\mathbb Z}}
\def\N{{\mathbb N}}
\def\R{{\mathbb R}}
\def\grad{\nabla}
\def\C{\mathcal C}
\def\P{\mathcal P}
\def\T{\mathcal T}
\def\I{\mathcal I}
\newcommand{\abs}[1]{\left| #1 \right|}
\newcommand{\inner}[1]{\langle #1 \rangle}
\newcommand{\norm}[1]{\left\lVert#1\right\rVert}
\newcommand{\spanvect}{\textnormal{span}}
\newcommand{\union}{\cup}
\newcommand{\Union}{\bigcup}
\def\intersect{\cap}
\def\Intersect{\bigcap}

%%% Page formatting
%\setlength{\headsep}{30pt}
\setlength{\parindent}{25pt}
\setlength{\textheight}{9in}

%%% Header and Footer Info
\pagestyle{fancy}
\fancyhead[L]{}
\fancyhead[C]{}
\fancyhead[R]{}
\fancyfoot[L]{}
\fancyfoot[C]{}
\fancyfoot[R]{\thepage\ of \pageref{LastPage}}
\renewcommand{\headrulewidth}{0pt}

\title{Binomial Coefficient Manipulations}
\author{Antonius Torode}
\date{November 25,2016}

%%% Document Starts now
\begin{document} 

\maketitle

To begin with, a useful idea is to sum each term of the binomial coefficient which will be used later. First, by definition of the binomial coefficient, we can write
\begin{align}
	{{n}\choose{k}}&=\frac{n!}{(n-k)!k!}\\
	&=\frac{(n-1)!n}{(n-k)!k!}\\&
	=\frac{(n-1)!n-k(n-1)!+k(n-1)!}{(n-k)!k!}\\
	&=\frac{(n-1)!(n-k)}{(n-k)!k!}+\frac{k(n-1)!}{(n-k)!k!}\\
	&=\frac{(n-1)!}{(n-1-k)!k!}+\frac{(n-1)!}{(n-k)!(k-1)!}\\&={{n-1}\choose{k}}+{{n-1}\choose{k-1}}\\
	&\equiv {{m}\choose{k}}+{{m}\choose{k-1}}, \textrm{ with }m=n-1.
\end{align} 
Observe for a moment that (6) can be expanding even further. By the same process of going from (1) to (1.6), we can say
\begin{align}
	{{n-1}\choose{k-1}} = {{n-2}\choose{k-1}}+{{n-2}\choose{k-2}}.
\end{align}
Thus, (1.6) becomes
\begin{align}
	{{n}\choose{k}}={{n-1}\choose{k}}+{{n-2}\choose{k-1}}+{{n-2}\choose{k-2}}.
\end{align}
If we continue this pattern, we can see that we can write the binomial coefficients as a sum of binomial coefficients with incrementally decreasing numerators (i.e n!, (n-1)!, (n-2)!,...). This gives
\begin{align}
	{{n}\choose{k}}&={{n-1}\choose{k}}+{{n-2}\choose{k-1}}+{{n-3}\choose{k-2}}+\cdots+{{1}\choose{k+2-n}}+{{0}\choose{k+1-n}} \\
	&=\sum_{i=1}^{n}{{n-i}\choose{k+1-i}} \\
	&=\sum_{i=0}^{n-1}{{n-1-i}\choose{k-i}}.
\end{align}
Note that we stop the above sequence when the numerator of our factorial sequence has reached zero. If we were to continue the sequence, we would end up having negative factorials in our numerator which would make evaluating the binomial coefficient at that term and the following terms difficult. However, if we happen to have a smaller $k$ than $n$, it may be such that we end up having negative $k$ numbers. This is ok for now, as it will lead to complex infinities in the denominator of our binomial coefficient expressions hence making those terms zero. This will be demonstrated later. Now, suppose we claim that for all $n\in \mathbb{N}$, that
\begin{align}
	\sum_{i=0}^{n}{{n}\choose{i}} =\sum_{i=0}^{n}\frac{n!}{(n-i)!i!}= 2^n.
\end{align}
We can then do a proof by induction to prove this is in fact true for all $n,k \in \mathbb{N}$. First, we can see checking the base cases hold as when $n=0$, we have $1=1$, when $n=1$, we have $2=2$, and when $n=3$ we have $4=4$. Next, let's assume for any $n=k$ that (1) holds true. Now, if we let $n=k+1$ we have from the left hand side of our expression,
\begin{align}
	\sum_{i=0}^{k+1}{{k+1}\choose{i}}&={{k+1}\choose{0}}+{{k+1}\choose{1}}+\cdots+{{k+1}\choose{k}}+{{k+1}\choose{k+1}} \\
	&={{k}\choose{0}}+{{k}\choose{-1}}+{{k}\choose{1}}+{{k}\choose{0}}+\cdots+{{k}\choose{k}}+{{k}\choose{k-1}}+{{k}\choose{k+1}}+{{k}\choose{k}}\\
	&=2{{k}\choose{0}}+2{{k}\choose{1}}+\cdots+2{{k}\choose{k-1}}+2{{k}\choose{k}}\\
	&=2\sum_{i=0}^{k}{{k}\choose{i}}\\
	&=2(2^k)\\
	&=2^{k+1}.
\end{align}
Therefore, we can see that for $n=k+1$ that equation (8) holds true, and thus we conclude by induction that (8) holds for all $n\in\mathbb{N}$. Note that in the above proof, we made use of ${{k}\choose{k+1}}={{k}\choose{-1}}=0$. If we were to evaluate each of these using the definition of the binomial coefficient above we may notice a slight issue. Suppose we try to evaluate ${{n}\choose{-1}}$. Using the definition from (1), we would have
\begin{align}
	{{n}\choose{-1}}&=\frac{n!}{(n-(-1))!(-1)!}\\&=\frac{n!}{(n+1)!(-1)!} \\
	&=\frac{n!}{(n)!(n+1)(-1)!} \\
	&=\frac{1}{(n+1)(-1)!}.
\end{align}
From above, we have a negative factorial in hte denominator of our expression. Since this is not easily determined as a positive integer factorial would be, we will need to expand this using the Gamma function. The Gamma function $\Gamma$ is defined by
\begin{align}
	\Gamma(n)=(n-1)!\equiv\int_{0}^{\infty}t^{n-1}e^{-t}\dt.
\end{align}
Using this gamma function with $n=0$ gives
\begin{align}
	\Gamma(0)&=\int_{0}^{\infty}t^{-1}e^{-t}\dt.
\end{align}
Since $\lim\limits_{t\rightarrow 0^+}t^{-1}e^{-t}=\infty$, we can say the integral under the curve from $0$ to $\infty$ will be divergent, and thus $\infty$. Therefore $\Gamma(0)\equiv\infty$. This allows us to write (18) as
\begin{align}
	\frac{1}{(n+1)(-1)!} = \frac{1}{(n+1)\Gamma(0)}=\lim\limits_{x\rightarrow \infty}\frac{1}{x}=0.
\end{align}
Thus, we can say that ${{n}\choose{-1}}=0$. Similarly, by the same process, if we have ${{n}\choose{n+1}}$ we get
\begin{align}
	{{n}\choose{n+1}}&=\frac{n!}{(n-(n+1))!(n+1)!}\\
	&=\frac{n!}{(-1)!(n)!(n+1)}\\
	&=\frac{1}{(-1)!(n+1)}\\
	&=0
\end{align}




%%%%%%%%%%%%%%%%%%%%%%%%%%%%%%%%%%%%%%%%%%%%%%%%%%%%%%%%%%%%%%%%%%%%%%%%%%%%%%%%%%%%%%%%%%%

%%%%%%%%%%%%%%%%%%%%%%%%%%%%%%%%%%%%%%%%%%%%%%%%%%%%%%%%%%%%%%%%%%%%%%%%%%%%%%%%%%%%%%%%%%%
\end{document}





















