\documentclass[11pt]{article}

%%% These are some packages that are useful
\usepackage{circuitikz} %used for circuit diagrams
\usepackage{lastpage} %allows us to determine how many total pages there are
\usepackage{amsfonts, lipsum}
\usepackage{amsmath,amssymb, amscd,amsbsy, amsthm, enumerate}
\usepackage{mdframed, titlesec, setspace,verbatim, multicol}
\usepackage[top=1in, bottom=1in, left=1in, right=1in]{geometry} %sets the margins
\usepackage[unicode]{hyperref} %enhanced references
\usepackage{tikz, pgfplots, xcolor}
\usepackage{fancyhdr} %creates a fancy header for labeling document/name/date 
\usepackage{listings}
\usepackage{xcolor}
\usepackage{vwcol}  
\usepackage{enumitem}

%%% Page formatting
%\setlength{\headsep}{30pt}
\setlength{\parindent}{25pt}
\setlength{\textheight}{9in}

%%% Header and Footer Info
\pagestyle{fancy}
\fancyhead[L]{}
\fancyhead[C]{}
\fancyhead[R]{}
\fancyfoot[L]{}
\fancyfoot[C]{}
\fancyfoot[R]{\thepage\ of \pageref{LastPage}}
\renewcommand{\headrulewidth}{0pt}

\title{Deriving The Speed Of Light In a Vacuum}
\author{Antonius Torode}

%%% Document Starts now
\begin{document} 

\maketitle

We begin with Maxwell's equations which follow in the MKS system of units \cite{Wolfram} as
\begin{multicols}{2}
	\noindent
	\begin{align}
	\nabla \cdot \vec{E} &= \frac{\rho}{\epsilon_0}\\
	\nabla \times \vec{E} &= -\frac{\partial \vec{B}}{\partial t} 
	\end{align}
	\begin{align}
	\nabla \cdot \vec{B} &= 0 \\
	\nabla \times \vec{B} &= \mu_0\vec{J}+\epsilon_0\mu_0\frac{\partial \vec{E}}{\partial t} 
	\end{align}
\end{multicols}
In the above equations, $t$ is time, $\epsilon_0$ is the permittivity of free space, $\mu_0$ is the permeability of free space, $\vec{E}$ is the electric field, $\vec{B}$ is the magnetic field, and $\vec{J}$ is the current density.  The definition of current is given by $\vec{I} \equiv \frac{dQ}{dt}\hat{I}$ which allows us to represent the current density as
\begin{align}
\vec{J} = \frac{d\vec{I}}{da_\perp} = \frac{d}{da_\perp}\left(\frac{dQ}{dt}\right)\hat{I}. \label{current density}
\end{align} 
Now, we can take the curl of equation (2) which gives
\begin{align}
\nabla \times (\nabla \times \vec{E}) = \nabla \times \left(-\frac{\partial \vec{B}}{\partial t}\right).
\end{align}
The left-hand side of this can be manipulated using the vector identity $\nabla \times (\nabla \times \vec{A}) = \nabla (\nabla \cdot \vec{A})- \nabla^2\vec{A}$. The right-hand side can also be rearranged since $\nabla$ is not an operation with respect to time which gives
\begin{align}
\nabla \times (\nabla \times \vec{E}) = \nabla (\nabla \cdot \vec{E})- \nabla^2\vec{E} = \nabla \times \left(-\frac{\partial \vec{B}}{\partial t}\right) = -\frac{\partial}{\partial t}(\nabla \times \vec{B}).
\end{align}
We can use equation (1) to write this as
\begin{align}
\nabla \left(\frac{\rho}{\epsilon}\right)- \nabla^2\vec{E} = -\frac{\partial}{\partial t}(\nabla \times \vec{B}).
\end{align}

If we assume we are in a perfect vacuum, then there would contain no matter and thus no charge. Therefore the charge density would be $\rho=0$. This implies the total charge is zero and thus $\frac{dQ}{dt}=0$. Hence, we can clearly see from equation (\ref{current density}) that within a vacuum $\vec{J}=0$. Using these results as well as (4), we can write equation (8) as
\begin{align}
-\nabla^2\vec{E} = -\frac{\partial}{\partial t}\left(\epsilon_0\mu_0\frac{\partial \vec{E}}{\partial t}\right) \implies \nabla^2\vec{E} = \epsilon_0\mu_0\frac{\partial^2 \vec{E}}{\partial t^2}.
\end{align}

This result in equation (9) can be recognized as the wave equation which is generally in the form
\begin{align}
\nabla^2\vec{\psi} = \frac{1}{v^2}\frac{\partial^2 \vec{\psi}}{\partial t^2},
\end{align}
where v is the velocity of a wave. Finally, if we think of the electric field $\vec{E}$ as a wave moving through a vacuum, then we can determine it's velocity by comparing equations (9) and (10). This is the speed of an electromagnetic wave (the speed of light) which gives
\begin{align}
v=c=\frac{1}{\sqrt{\epsilon_0\mu_0}}.
\end{align}
We can approximate this based on the values of $\epsilon_0 = 8.85418782 \times 10^{-12}  s^4 A^2/(m^3 kg)$ and $\mu_0 =4\pi \times 10^{-7} Wm$ which gives
\begin{align}
c=\frac{1}{\sqrt{\epsilon_0\mu_0}} = \frac{1}{\sqrt{[8.85418782 \times 10^{-12}  s^4 A^2/(m^3 kg)][4\pi \times 10^{-7} Wm]}} = \boxed{299,792,458 \frac{m}{s}}.
\end{align}


\begin{thebibliography}{9}
	\bibitem{Wolfram} ``Wolfram MathWorld: The Web's Most Extensive Mathematics Resource." Wolfram MathWorld. N.p., n.d. Web. 
\end{thebibliography}

%%%%%%%%%%%%%%%%%%%%%%%%%%%%%%%%%%%%%%%%%%%%%%%%%%%%%%%%%%%%%%%%%%%%%%%%%%%%%%%%%%%%%%%%%%%

%%%%%%%%%%%%%%%%%%%%%%%%%%%%%%%%%%%%%%%%%%%%%%%%%%%%%%%%%%%%%%%%%%%%%%%%%%%%%%%%%%%%%%%%%%%
\end{document}





















