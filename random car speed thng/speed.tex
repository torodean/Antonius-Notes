\documentclass[]{article}
\usepackage{amsfonts, lipsum}
\usepackage{amsmath,amssymb, amscd,amsbsy, amsthm, enumerate}
\usepackage{mdframed, titlesec, setspace,verbatim, multicol}
\usepackage[top=1in, bottom=1in, left=1in, right=.45in]{geometry}
\usepackage[unicode]{hyperref}
\usepackage{wasysym}
\usepackage[nottoc]{tocbibind}
\usepackage{tikz, fancyhdr, polynom}
\usepackage{xcolor}
\usepackage{pdfpages}
%opening
\title{How fast was the car passing us going?}
\author{Antonius Torode}
\date{\today}

\begin{document}

\maketitle

\begin{abstract}
This is the derivation of a formula for determining how fast a car passed us at. We neglect any relativistic effects as cars in general cannot travel near the speed of light.
\end{abstract}

\section{The derivation without relativistic effects}
To begin, suppose you are traveling at some velocity in a straight line down a flat road $v_1$. Assume that you are traveling at a constant velocity and your acceleration is zero and remains zero. Now, assume that another car, also with a zero acceleration passes you are some speed $v_2$. We can start with the distance formula
\begin{align}
	velocity = \frac{distance}{time} \Longleftrightarrow time = \frac{distance}{velocity}. 
\end{align}
Note that $v_1$ and $v_2$ are measured relative to the stationary road. That is, the speeds are measured as $v_1$ and $v_2$ only by someone whom is moving at the same speed the road is. This does not help us, so instead let's look at these speeds in the frame of within a car. If we are within car 1, the speed of the car relative to us is zero. That is $v_1'=0$. Similarly, the speed of the other car measured from our frame is the speed of the other car we observe from ours as $v_2'=v_2-v_1$ and so solving for $v_2$ we get 
\begin{align}
v_2=v_2'+v_1.
\end{align} Now, while within car one, we cannot necessarily measure the speed of the car two in our frame directly. We can however, measure the time it takes for the front end of that car to pass from the back of our car to the front of our own car. Call this time $t$. Now, the distance from the front of our car to the back is $\Delta d$, and so if the car passes our car from back to front in some time $t$, we can determine by equation (1) that it was traveling at speed
\begin{align}
v_2'=\frac{\Delta d}{t}.
\end{align}
And thus, the speed of the other car relative to the road is then
\begin{align}
v_2=\frac{\Delta d}{t}+v_1.
\end{align}
\section{Units}
It is generally approprate to measure speeds of cars in Miles per hour (in The United States), lengths of something such as a car in feet, and times such as cars passing eachother in seconds. If we use these units, we can see that (4) becomes
\begin{align}
v_2=\frac{\Delta d \times \left(\frac{1 mi}{5280 ft}\right)}{t\left(\frac{1 hr}{3600 s}\right)}+v_1.
\end{align}
And now, plugging in units of seconds for $t$, feet for $\Delta d$, and miles per hour for $v_1$ will give a speed of the second car in miles per hour.

\section{Example}
Suppose you are traveling 60 miles per hour in a 20 foot truck and another car passes you in 1 second. Then by equation (5) the other car was traveling at
\begin{align}
v_2=\frac{20 ft \times \left(\frac{1 mi}{5280 ft}\right)}{1 s\left(\frac{1 hr}{3600 s}\right)}+60 \frac{mi}{hr} = \frac{810}{11} \frac{mi}{hr} \approx 73.6 \frac{mi}{hr}.
\end{align}

%\section{Including special relativistic effects}





\end{document}
